\documentclass[MasterThesisMain.tex]{subfiles}
\begin{document}
	\chapter{Introduction}
	
In December $2017$ and May $2018$, I was asked to partake in Grazing-incidence small-angle scattering(GISAXS) experiments investigating the structure of thin films at The Cornell High Energy Synchrotron Source (CHESS). A thin film is a thin layer of polymer deposited onto a wafer, in this case a silicon wafer. The experiments were conducted with the aim of understanding the reorganisation of star-block polymers on a silicon wafer when exposed to vapour annealing. GISAXS maps were was used to study how the star-block polymers arranged themselves when the volume of the polymer increased. GISAXS maps gives the researcher a snapshot of the polymers structure at that point in time. The GISAXS maps can be used to calculate the thickness of the polymer, this is not a trivial since it involves fitting a model[explain a bit more], it is also done post experiment and is time consuming. 

This leads to the experimental method called Spectroscopic Reflectometry, and the meaning of these two words gives insight into the nature of the method. Spectroscopic is a study of the interaction between electromagnetic radiation and matter \cite{levinechemistry} and reflectometry is the study of an object using reflected light. This experimental technique can be used to measure the thickness of a thin film. The idea of running the two experimental technique parallel at CHESS was to monitor the thin films in-situ giving us the ability to explore different solvent vapour annealing(SVA) protocols and see how the structure changed during these. The thickness measurements done at CHESS using Spectroscopic Reflectometry were not successful since the software's thin film model and the reflectance data did not fit well. Since beam time is precious, this problem was to be investigated back at Roskilde University in between beam times.

Spectroscopic Reflectometry is not the only experimental technique available that can measure thin film thickness, there is Atomic force microscopy, Ellipsometry and X-ray reflectometry. Spectroscopic reflectometry has been chosen to complement GISAXS because it can be used in-situ and used in harsh conditions. The size of the apparatus plays a huge role as the free space in a synchrotron hutch is limited. It is a primitive technique compared to the other techniques that use electromagnetic radiation, and it is important investigate how far this technique can be pushed.

This thesis has two purposes, it is a study of how best to use the third party apparatus called NanoCalc XR spectrometer and software made by the company Ocean Optics, (\url{https://oceanoptics.com/}) and an investigation into how the reflectance measurements change during the solvent vapour annealing of homopolymers and copolymers.

The first purpose provides a experimental fundament since the NanoCalc XR spectrometer and software will be used to investigate the thin films and this thesis will serve as an introduction to the apparatus and how light reflects and transmits through the thin film since the user manual for the spectrometer and software is lacking in an explanation of the theory behind the thin film modelling and fitting.

The second purpose is the investigate how the reflectance curves change during solvent vapour annealing and is it possible to infer what is happening to the thin film during this process. This will be usefully for experiments conducted at synchrotron sources as the promising swelling protocols can be used.     

\section{Research Question}
What are the advantages and limitations when using optical spectral reflectance for determining the thickness of thin polymer films during solvent vapour annealing? [NHB - Not too sure about this research question, i have not though about the advantages and limitations.] 

What is the optimal modelling and fitting method for the optical spectral reflectance measurements and thickness determination of homopolymers deposed on thin films during solvent vapour annealing?  
		
Can the same thickness determination be used on thin films with a nano scale structure such as the block-copolymer Polystyrene-b-Polyisoprene?


%Notes
%develop model of homopolyer thin films --> can this be used to for thickness measurements of thin films with nano scale structure namely block co polymers PS-b-PI.	

\section{Structure of this thesis} 
The structure of this thesis is as follows. Chapter \ref{lighttheory} will introduce the basic theory of light propagation through a vacuum and through a transparent medium. It will touch on the complex refractive index and the Fresnel equations used to model light reflecting and transmitting in the thin films. Chapter \ref{experimentalmethod} will introduce the polymers, creation of the thin films, the Nanocalc spectrometer and how the measurements are taken.          
\end{document}