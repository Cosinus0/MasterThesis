\documentclass[MasterThesisMain.tex]{subfiles}
\begin{document}
	\chapter{Introduction}
	
In December $2017$ and May $2018$, I was asked to partake in grazing-incidence small-angle scattering(GISAXS) experiments investigating the structure of thin films at The Cornell High Energy Synchrotron Source (CHESS). A thin film is a thin layer on the nanoscale of polymer deposited onto a wafer, in this case a silicon wafer. The experiments were conducted with the aim of understanding the reorganisation of star-block polymers on a silicon wafer when exposed to vapour annealing. GISAXS maps are used to study how the star-block polymers arranged themselves when the polymer film swells. The swelling dilutes the polymer, lowering the glass transition temperature of the polymers, reducing the viscosity and increasing the chain mobility of the polymer. For diblock copolymers, swelling reduces the interfacial tension between the blocks of the diblock copolymer, and between the polymer blocks and the substrate \cite{posseltintro}. GISAXS maps gives the researcher a snapshot of the polymers structure at that point in time. The GISAXS maps can be used to calculate the thickness of the polymer, but fitting a model onto the data is non-trivial and post experiment.

This leads to the used of the experimental method called optical spectral reflectance to measure the thickness of a thin film. Using optical spectral reflectance parallel to GISAXS thin films thickness can be monitored in-situ during the solvent vapour annealing(SVA) protocols. The thickness measurements done at CHESS using the optical spectral reflectance were not optimal since the software's thin film model and the reflectance data did not fit well. Since beam time is precious, this problem was to be investigated back at Roskilde University in between beam times.

Optical spectral reflectance is not the only experimental technique available that can measure thin film thickness, there is Atomic force microscopy, Ellipsometry and X-ray reflectometry. Optical spectral reflectance has been chosen to complement GISAXS because it can be used in-situ and in harsh conditions. The size of the apparatus plays a huge role as the free space in a synchrotron hutch is limited. It is a primitive technique compared to the other techniques that use electromagnetic radiation, and it is important to investigate how far this technique can be pushed.

This thesis has two purposes. The first purpose is to describe the optical spectral reflectance experimental method and how the NanoCalc XR spectrometer and software made by the company Ocean Optics, (\url{https://oceanoptics.com/}) is used and can the use be optimised for the polymers used in this thesis. This will serve as an introduction to the apparatus and how light reflects and transmits through the thin film. The second purpose is to investigate how the reflectance measurements change during the solvent vapour annealing of homopolymers and diblock copolymers and if it is possible to infer the thickness of the thin films during the solvent vapour annealing. 

\section{Research Question}
What are the advantages and limitations when using optical spectral reflectance for determining the thickness of thin polymer films during solvent vapour annealing?

What is the optimal modelling and fitting method for the optical spectral reflectance measurements and thickness determination of the homopolymers, polystyrene and polyisoprene thin films during solvent vapour annealing?  
		
Can the same thickness determination be used on thin films with a horizontal nano scale structure such as the diblock-copolymer Polystyrene-b-Polyisoprene?


\section{Structure of this thesis} 
The structure of this thesis is as follows. Chapter \ref{lighttheory} will introduce the basic theory of light propagation through a vacuum and through a transparent medium. It will touch on the complex refractive index and the Fresnel equations used to model the reflectance and transmittance of light illuminating the thin films. Chapter \ref{experimentalmethod} will introduce the polymers, creation of the thin films, the Nanocalc spectrometer and how the measurements are taken.  

        
\end{document}