\documentclass[MasterThesisMain.tex]{subfiles}
\begin{document}
	\chapter{Introduction}
	
In December $2017$ and May $2018$ I was asked to partake in Grazing-incidence small-angle scattering(GISAXS) experiments of thin films at The Cornell High Energy Synchrotron Source (CHESS). A thin film is a thin layer of polymer deposited onto a wafer, in this case a silicon wafer. The experiments were with regards to research in structure evolution of star-block polymers on a silicon wafer when exposed to vapour annealing. X-ray scattering reflectometry was used to study how the star-block polymers arranged themselves when the volume of the polymer increased. X-ray reflectometry gives the researcher a snapshot of the study at that point in time. These snapshots can be used to calculate the thickness of the polymer, but this can only be done post experiment and is time consuming. This leads to the experimental method called Spectroscopic Reflectometry. Spectroscopic a study of the interaction between light and matter and reflectometry the characterisation of an object using reflected light. This experimental technique can be used to measure the thickness of a thin film. The idea of running the two experimental technique parallel at CHESS was to monitor the thin films in real time giving us the ability to explore different vapour annealing protocols and see how the structure changed.          
\end{document}