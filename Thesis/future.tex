\documentclass[MasterThesisMain.tex]{subfiles}
\begin{document}

\chapter{Suggestions for Future Research}
At the conclusion of this master's thesis, other research questions have presented themselves which I deem to be excellent research questions for project work. 

The fitting protocol implemented in this thesis could fit for only one thin film layer. Diblock copolymers and other exotic polymers show repeating structure which call for more complex models and better fitting implementations. I deem this to be a fruitful project idea for students studying either mathematics, physics or computer science, as a careful study of a polymer during solvent vapour annealing using optical spectral reflectance can complement GISAXS measurements. The measurements collected by myself can easily be reused and reanalysed.    

I have discussed that the absorption of light by polymers could be why I see a decrease in the reflectance measurements during the solvent vapour annealing. This research question could be tackled by both physics student and/or chemistry students.  

The fitting of the reflectance measurement by my fitting implementation has hinted at an increase of the refractive index of the thin film. This could be studied further with in-situ ellipsometry. Having a clear understanding of how the refractive index of the polymers change during the solvent vapour annealing can greatly help with the modelling aspect as there is one less variable to fit to. In-situ ellipsometry can also investigate the refractive index dispersion during the solvent vapour annealing. This research question can be studied by students of either physics or chemistry. 

Modelling the vapour pressure inside the solvent vapour annealing chamber and the uptake of solvent by thin films could be an excellent mathematics research question. This can give insight into why the thin film swells slowly but dries fast.


\end{document}