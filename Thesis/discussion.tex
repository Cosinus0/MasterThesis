\documentclass[MasterThesisMain.tex]{subfiles}
\begin{document}
\chapter{Discussion}

\section{Experimental setup}
The experimental setup can be used in two ways. The first way with the optical fibre in the single point stage and the second way in the solvent vapour annealing chamber. When the optical fibre is fitted into the solvent vapour annealing chamber, the light must pass through a sapphire lens. This causes a great deal of reflected of light and effects the reflectance measurements taken. This can be seen by holding the static measurements seen in table \ref{tab:polymers} against the first fitted thickness values found in the figures for each polymer in chapter \ref{ch:results}. The effects may be small at the beginning of the solvent vapour annealing when the polymer has not swelled, but during swelling the reflectance measurement drops and can be zero in wavelength intervals. The sapphire lens effectively increases the dark measurement and when calculating the reflectance, since the dark measurement is in the denominator, lowers the reflectance measurements. If the thin film measurement decreases during the swelling this will also impact the reflectance measurement, lowering it even more. The dark measurements are taken when the optical fibre is fitted into the solvent vapour annealing chamber with a piece of black fabric where the wafer would lay. The black fabric is a piece of black cotton which has been taken from a bit of clothing. The effect, if there is any, of the black fabric has not been investigated. The distance from the wafer to the optical fiber is important, as it is seen to shift the whole reflectance measurement up and down the y axis. Up if the distance increases and down if it decreases. It is therefore very important to use a step wafer where the thickness is known to adjust the single point stage before a static measurement of a wafer with a thin film. When the optical fiber is placed in the solvent vapour annealing chamber the distance between the optical fiber and the wafer is much greater than it intended thus the sapphire lens is used to focus the light to illuminate the wafer. Calibrating the solvent vapour annealing setup is impossible at this point since the chamber is not big enough for the step wafer to be place in the chamber.

\section{Dispersion of polymers}



\section{Thickness fitting}

\section{Solvent vapour annealing}

\section{Results}

\subsection{Polystyrene vs. Polyisoprene}

\subsection{Polystyrene/Polyisoprene vs. polystyrene-b-polyisoprene}

\end{document}