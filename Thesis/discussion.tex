\documentclass[MasterThesisMain.tex]{subfiles}
\begin{document}
\chapter{Discussion}

\section{Experimental setup}
The experimental setup can be used in two ways, with the optical fibre in the single point stage or in the solvent vapour annealing chamber. When the optical fibre is fitted into the solvent vapour annealing chamber, the light must pass through the a sapphire lens. This causes a great deal of reflected of light and effects the reflectance measurements taken. This can be seen by holding the static measurements seen in table \ref{tab:polymers} against the first fitted thickness values found in the figures for each polymer in chapter \ref{ch:results}. The effects may be small at the beginning of the solvent vapour annealing when the polymer has not swelled, but during swelling the reflectance measurement drops and can be zero in intervals. The sapphire lens effectively increases the dark measurement and when calculating the reflectance, since the dark measurement is in the denominator, lowers the reflectance measurements. If the thin film measurement decreases during the swelling this will also impact the reflectance measurement lowering it even more. The dark measurements are taken when the optical fibre is fitted into the solvent vapour annealing chamber with a piece of black fabric where the wafer would lay. 

\end{document}