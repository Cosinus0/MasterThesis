\documentclass[MasterThesisMain.tex]{subfiles}
\begin{document}
\chapter{Conclusion} \label{ch:conclusion}
In the course of this thesis using the optical spectral reflectance experimental technique, the goals have been to investigate the advantages and limitations of the experimental technique when determining the thickness of thin polymer films during solvent vapour annealing, investigation of optimal modelling and fitting method to determine thickness of homopolymers during solvent vapour annealing and an investigation into whether the same thickness determination could be used to determine thickness of a thin film such as the diblock copolymer polystyrene-b-polyisoprene. The advantages of this experimental technique is that the modelling aspect uses the fresnel equations and are easily calculated. These are easy to understand and easy to model multiply layers since the fresnel equations are recursive. The limitations of this technique is that the dark measurements can greatly impact the reflectance measurements of a thin films. Care is need when performing a dark measurement and a proper technique is need for the solvent vapour annealing chamber as there is light reflected back through the optical fiber due to a focusing lens. The modelling of the homopolymers polystyrene and polyisoprene has used the fresnel equation shown in equation \ref{eq:multilayer}. This model assumes that the homopolymers consist of one layer and has a homogeneous thickness and real refractive index. The fitting implemented is a script finding the lowest mean square error for each reflectance measurement. From the result it can be seen that the thickness determination can be found during the solvent vapour annealing when modelling the layer as a homogeneous thickness with real refractive indices. Using the same model to determine the thickness of the diblock copolymer polystyrene-b-polyisoprene shows strange results that does not reflect the solvent vapour annealing process. The modelling and fitting done in this thesis seems optimal for homopolymers lacking structure but fails when applied to polymer systems with structure.     

\end{document}