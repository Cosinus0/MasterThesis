% Document Packages
\documentclass[12pt,a4paper]{report}
\usepackage[T1]{fontenc} 
\usepackage[utf8]{inputenc}
\usepackage[english]{babel}
\usepackage{xparse}
\usepackage{subfiles}
\usepackage{url}
\usepackage{titling}
\usepackage{fancyhdr,ragged2e}
\pagestyle{fancy}
\fancyhead[LE,RO]{\itshape \nouppercase \rightmark}
\fancyhead[LO,RE]{\itshape \nouppercase Chapter \arabic{chapter}}

%Math Packages
\usepackage{amsmath}
\usepackage{amsthm}
\usepackage{amssymb}
\usepackage{amsfonts}
\usepackage{physics}
\usepackage{mathtools}
\usepackage{siunitx}

\usepackage{listings}
\lstset{inputpath=appendixscripts/}
\usepackage[framed,numbered,autolinebreaks,useliterate]{mcode}
\usepackage{color} %red, green, blue, yellow, cyan, magenta, black, white
\definecolor{mygreen}{RGB}{28,172,0} % color values Red, Green, Blue
\definecolor{mylilas}{RGB}{170,55,241}

%Figure Packages
\usepackage{graphicx}
\graphicspath{{figures/}}
\usepackage{float}
\usepackage{wrapfig}
\usepackage{subfigure}
%\usepackage{subcaption}
\usepackage{tikz}
\usepackage{pgfplots}
\pgfplotsset{compat=1.5}
\pgfplotsset{samples=200}

%Chemistry
\usepackage{chemfig}
\usepackage[version=3]{mhchem}

%Page layout Packages
\usepackage[a4paper,width=150mm,top=25mm,bottom=25mm]{geometry}

%Misc
\usepackage{marginnote}
\usepackage{url}

%Newcommands
\renewcommand*{\marginfont}{\footnotesize}
\newcommand{\subtitle}[1]{
  \posttitle{
    \par\end{center}
    \begin{center}\large#1\end{center}
    \vskip0.5em}
}
\DeclareSIUnit\atmos{atm}

%Bib Packages
\usepackage[numbib,nottoc]{tocbibind}
\usepackage[square,sort,comma,numbers]{natbib}






\author{Nathan Hugh Barr}
\title{}
\subtitle{}
\linespread{1.3}
\begin{document}
\maketitle

\begin{abstract}

In this thesis, the solvent vapour annealing of the homopolymers polystyrene and polyisoprene thin films and the diblock copolymer polystyrene-b-polyisoprene thin film have been investigated using the experimental technique optical spectral reflectance and the thickness of the thin films have been modelled using the Fresnel equations. The layered model consists of an ambient with a refractive index, a homogeneous thin film with a refractive index and thickness, a silicon oxide layer fixed at $\SI{2}{\nano\meter}$ and a silicon wafer. Both the silicon oxide layer and silicon wafer have refractive indices and have been taken from Ocean Optics Nano-Calc software \cite{nanocalcmanual}.  A fitting protocol has been implemented using the mean square error and three values are fitted per measurement, the refractive index for the ambient, the refractive index for the thin film and the thickness of the thin film. The modelling and fitting for the homopolymers polystyrene and polyioprene seem optimal whereas when applied to the polystyrene-b-polyisoprene thin film the modelling is suboptimal.  

\end{abstract}

\renewcommand{\abstractname}{Acknowledgements}
\begin{abstract}
This master thesis started in the spring of 2018 and submitted in April of 2019. This thesis is the final written project of the mathematics and physics master's degree at Roskilde University. There is a number of people i would like to thank, as their support has helped me immensely.

I would like to begin by thanking my thesis supervisor Dorthe Posselt for the opportunity to help with her research at the Cornell High Energy Synchrotron Source and her input with regards to this thesis. I would like to thank Bo Jakobsen for his help with the experimental setup and his front end data collecting program which was written to be used with the spectrometer. I would like to thank Sina Ariaee for his help when i ran into problems with solvent vapour annealing experiments and with any other problems that followed.

I would like to thank Esben Thormann and Saeed Zajforoushan Moghaddam from the Thormann research group located at the Technical University of Denmark (DTU) for access to there ellipsometry machine and the time they used to help me collect data on the thin film wafers used in this thesis.

I would like to thank Daniel Olesen Fejerskov for reading through my thesis and for his usefully comments and suggestions.

I would like to thank the mathematics and physics department(IMFUFA) for creating a wonderful learning environment that has shaped my personal growth through the past six years.     

\end{abstract}

\tableofcontents

\subfile{introduction.tex}
\subfile{theory.tex}
\subfile{method.tex}
\subfile{analysis.tex}
\subfile{discussion.tex}
\subfile{conclusion.tex}
\subfile{future.tex}

\newpage
\bibliography{reference}
\bibliographystyle{unsrt}

\subfile{appendix.tex}
\end{document}